\documentclass[a4paper,12pt,twocolumn]{article}
\usepackage{picinpar,graphicx}
	\usepackage{indentfirst}
		\usepackage{cite}
			\usepackage{multirow}
				\usepackage{CJK}
				
	\setlength{\parindent}{3em}
\setlength{\parskip}{1em}
\pagestyle{plain}
\linespread{1.5}
\begin{document}
\begin{center}
	
	\textbf{\bfseries \LARGE Integration of Artificial Intelligence and Learning Education} 
\end{center}
\begin{center}
	Lina Zhu
\end{center}
\begin{center}
	\today
\end{center}

	\par 	This article describes the adaptive education system in the e-learning platform, which aims to respond to each learner's different learning processes. E-learning services and materials are tailored to provide good adaptive learning. This type of educational approach attempts to understand and detect the individual’s ability to meet specific needs in the learning environment and to use appropriate learning processes and the required expertise in the learning process.
	\par Therefore, accurate student profiles and models are created by analyzing their emotional state, knowledge level, and their individual personality characteristics and skills. The data obtained can be effectively utilized and utilized to develop an adaptive learning environment. Once acquired, these learner models can be used in two ways. The first is the pedagogy that informs experts and designers of adaptive education systems. The second is to provide the system with dynamic self-learning skills from the behaviors displayed by teachers and students to create appropriate pedagogical methods and automatically adjust the e-learning environment to suit the teaching method.
	\par In this regard, Just like figure.~\ref{pic1}artificial intelligence technology may have several reasons, including their ability to develop and imitate human reasoning and decision-making processes, and minimize the sources of uncertainty in order to achieve an effective learning and teaching environment. These learning capabilities ensure that learners and systems improve their lifelong learning mechanisms. It is a technology developed to improve the child's learning situation, which can effectively enhance children's interest in learning to teach students in accordance with their aptitude.
	\footnote{from"A Survey of Artificial Intelligence Techniques Employed for   Adaptive Educational  Systems Within E-Learning Platform	"} 

\begin{figure}[htp]
	\centering
	\includegraphics[width=9cm]{figure1.jpg}
	\caption{\bfseries{  Effective Integration of Artificial Intelligence and Education Systems}}\label{pic1}
\end{figure}


%\footnote{from"Network newspaper"} 
	\begin{thebibliography}{99}
%	\bibitem{pa}   Newell A, Simon H A. Human Problem Solving. Englewood Cliffs, NJ: Prentice-Hall, 1972 
%	\bibitem{pa}  Minsky M L. The Society of Mind. New York: Simon and Schuster, 1986 
	\bibitem{pa}  T.Kidd,OnlineEducationandAdultLearning.New York: Hershey, 2010.
	\bibitem{pa}    Khalid Almohammadi \emph{etal.} A Survey of Artificial Intelligence Techniques Employed for  Adaptive Educational Systems Within E-Learning Platform, [J] .2017,46--64.
\end{thebibliography}
\end{document}